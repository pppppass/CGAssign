% !TeX encoding = UTF-8
% !TeX program = LuaLaTeX
% !TeX spellcheck = en_US

% Author : Zhihan Li
% Description : Report of Assignment 3

\documentclass[english, nochinese]{../textmpls/pkupaper}

\usepackage[paper, upgreek]{../textmpls/def}

\newcommand{\cuniversity}{Peking University}
\newcommand{\cthesisname}{Computer Graphics}
\newcommand{\titlemark}{Report of Assignment 3}

\DeclareRobustCommand{\authoring}%
{%
\begin{tabular}{c}%
Zhihan Li \\%
1600010653%
\end{tabular}%
}

\title{\titlemark}
\author{\authoring}

\begin{document}

\maketitle

\begin{thmquestion}
\ 
\begin{thmanswer}
Consider the shear matrix
\begin{equation}
\widetilde{\mathbf{S}} = \msbr{ 1 & u \\ 0 & 1 }
\end{equation}
first, where $ u = \cot \theta $. Using singular eigenvalue decomposition, we may obtain
\begin{equation}
\widetilde{\mathbf{S}} = \widetilde{\mathbf{P}} \widetilde{\mathbf{D}} \widetilde{\mathbf{Q}}^{\rmut},
\end{equation}
where $\widetilde{\mathbf{P}}$ and $\widetilde{\mathbf{Q}}$ are orthogonal and $\widetilde{\mathbf{D}}$ is non-negative diagonal. Explicitly, a choice for the matrices is
\begin{align}
\widetilde{\mathbf{P}} &= \left[\begin{matrix}- \frac{2}{\sqrt{\left(u + \sqrt{u^{2} + 4}\right)^{2} + 4}} & - \frac{u + \sqrt{u^{2} + 4}}{\sqrt{\left(u + \sqrt{u^{2} + 4}\right)^{2} + 4}}\\\frac{u + \sqrt{u^{2} + 4}}{\sqrt{\left(u + \sqrt{u^{2} + 4}\right)^{2} + 4}} & - \frac{2}{\sqrt{\left(u + \sqrt{u^{2} + 4}\right)^{2} + 4}}\end{matrix}\right], \\
\widetilde{\mathbf{Q}} &= \left[\begin{matrix}- \frac{u + \sqrt{u^{2} + 4}}{\sqrt{\left(u + \sqrt{u^{2} + 4}\right)^{2} + 4}} & - \frac{2}{\sqrt{\left(u + \sqrt{u^{2} + 4}\right)^{2} + 4}}\\\frac{2}{\sqrt{\left(u + \sqrt{u^{2} + 4}\right)^{2} + 4}} & - \frac{u + \sqrt{u^{2} + 4}}{\sqrt{\left(u + \sqrt{u^{2} + 4}\right)^{2} + 4}}\end{matrix}\right], \\
\widetilde{\mathbf{D}} &= \left[\begin{matrix}\frac{4 \sqrt{u^{2} + 4}}{\left(u + \sqrt{u^{2} + 4}\right)^{2} + 4} & 0\\0 & \frac{u^{3} + u^{2} \sqrt{u^{2} + 4} + 4 u + 2 \sqrt{u^{2} + 4}}{u^{2} + u \sqrt{u^{2} + 4} + 4}\end{matrix}\right],
\end{align}
where $ \det \widetilde{\mathbf{P}} = \det \widetilde{\mathbf{Q}} = 1 $. Because $\widetilde{\mathbf{P}}$ and $\widetilde{\mathbf{Q}}^{\rmut}$ are orthogonal with determinant $1$, therefore they represent rotations. Because $\widetilde{\mathbf{D}}$ is positive diagonal, therefore it stands for scaling transform. Consequently, the original scaling matrix can be decomposed into
\begin{equation}
\mathbf{S} = \msbr{ \widetilde{\mathbf{S}} & 0 & 0 \\ 0 & 1 & 0 \\ & & 1 } = \msbr{ \widetilde{\mathbf{P}} & 0 & 0 \\ 0 & 1 & 0 \\ & & 1 } \msbr{ \widetilde{\mathbf{D}} & 0 & 0 \\ 0 & 1 & 0 \\ & & 1 } \msbr{ \widetilde{\mathbf{Q}}^{\rmut} & 0 & 0 \\ 0 & 1 & 0 \\ & & 1 },
\end{equation}
where the last three matrices represent rotation, translation, rotation respectively. Symbolic verification can be seen in \verb"Problem1.ipynb".
\end{thmanswer}
\end{thmquestion}

\begin{thmquestion}
\
\begin{thmanswer}
The composed matrix is
\begin{equation}
\begin{split}
\mathbf{R} &= \mathbf{R}_x \rbr{\theta_x} \mathbf{R}_y \rbr{\theta_y} \mathbf{R}_z \rbr{\theta_z} \\
&= \msbr{ 1 & 0 & 0 & 0 \\ 0 & \cos \theta_x & -\sin \theta_x & 0 \\ 0 & \sin \theta_x & \cos \theta_x & 0 \\ & & & 1 } \msbr{ \cos \theta_y & 0 & \sin \theta_y & 0 \\ 0 & 1 & 0 & 0 \\ -\sin \theta_y & 0 & \cos \theta_y & 0 \\ & & & 1 } \msbr{ \cos \theta_z & -\sin \theta_z & 0 & 0 \\ \sin \theta_z & \cos \theta_z & 0 & 0 \\ 0 & 0 & 1 & 0 \\ & & & 1 } \\
&= \msbr{ \cos \theta_y & 0 & \sin \theta_y & 0 \\ \sin \theta_x \sin \theta_y & \cos \theta_x & -\sin \theta_x \cos \theta_y & 0 \\ -\cos \theta_x \sin \theta_y & \sin \theta_x & \cos \theta_x \cos \theta_y & 0 \\ & & & 1 } \msbr{ \cos \theta_z & - \sin \theta_z & 0 & 0 \\ \sin \theta_z & \cos \theta_z & 0 & 0 \\ 0 & 0 & 1 & 0 \\ & & & 1 } \\
&= \msbr{ \cos \theta_y \cos \theta_z & -\cos \theta_y \sin \theta_z & \sin \theta_y & 0 \\ \sin \theta_x \sin \theta_y \cos \theta_z + \cos \theta_x \sin \theta_z & -\sin \theta_x \sin \theta_y \sin \theta_z + \cos \theta_x \cos \theta_z & -\sin \theta_x \cos \theta_y & 0 \\ -\cos \theta_x \sin \theta_y \cos \theta_z + \sin \theta_x \sin \theta_z & \cos \theta_x \sin \theta_y \sin \theta_z + \sin \theta_x \cos \theta_z & \cos \theta_x \cos \theta_y & 0 \\ & & & 1 }.
\end{split}
\end{equation}
\end{thmanswer}
\end{thmquestion}

\begin{thmquestion}
\ 
\begin{thmanswer}
This is possible. For example, we try to devise translation matrix $\mathbf{T}'$, such that
\begin{equation} \label{Eq:Equ}
\mathbf{M} = \mathbf{T} \mathbf{R} \mathbf{S} = \mathbf{R} \mathbf{S} \mathbf{T}'.
\end{equation}
Assume
\begin{gather}
\mathbf{R} = \msbr{ \widetilde{\mathbf{R}} & 0 \\ & 1 }, \mathbf{S} = \msbr{ \widetilde{\mathbf{S}} & 0 \\ & 1 }, \\
\mathbf{T} = \msbr{ \mathbf{I} & \mathbf{x} \\ & 1 }, \mathbf{T}' = \msbr{ \mathbf{I} & \mathbf{x}' \\ & 1}.
\end{gather}
From \eqref{Eq:Equ}, we deduce that
\begin{equation}
\msbr{ \widetilde{\mathbf{R}} \widetilde{\mathbf{S}} & \mathbf{x} \\ & 1 } = \msbr{ \widetilde{\mathbf{R}} \widetilde{\mathbf{S}} & \widetilde{\mathbf{R}} \widetilde{\mathbf{S}} \mathbf{x}' \\ & 1 }.
\end{equation}
Since $\mathbf{R}$ and $\mathbf{S}$ are invertible and so are $\widetilde{\mathbf{R}}$ and $\widetilde{\mathbf{S}}$, therefore if we let
\begin{equation}
\mathbf{x'} = \widetilde{\mathbf{S}}^{-1} \widetilde{\mathbf{R}}^{-1} \mathbf{x},
\end{equation}
then $\mathbf{T}'$ is a appropriate translation matrix and \eqref{Eq:Equ} is satisfied.

In conclusion, different order of transformations (with some transformations modified) may yield the same result.

(More precisely, because
\begin{gather}
\msbr{ \mathbf{I} & \mathbf{x} \\ & 1 } \msbr{ \widetilde{\mathbf{R}} & 0 \\ & 1 } = \msbr{ \widetilde{\mathbf{R}} & 0 \\ & 1 } \msbr{ \mathbf{I} & \widetilde{\mathbf{R}}^{-1} \mathbf{x} \\ & 1 }, \\
\msbr{ \mathbf{I} & \mathbf{x} \\ & 1 } \msbr{ \widetilde{\mathbf{S}} & 0 \\ & 1 } = \msbr{ \widetilde{\mathbf{S}} & 0 \\ & 1 } \msbr{ \mathbf{I} & \widetilde{\mathbf{S}}^{-1} \mathbf{x} \\ & 1 }, \\
\end{gather}
therefore rotation, and scaling as well, may be exchangable with a translation (with some modification to the trnaslation). Therefore, compositions $ \mathbf{T} \mathbf{R} \mathbf{S} $, $ \mathbf{R} \mathbf{T} \mathbf{S} $ and $ \mathbf{R} \mathbf{S} \mathbf{T} $ have identical ranges. Moreover, $ \mathbf{T} \mathbf{S} \mathbf{R} $, $ \mathbf{S} \mathbf{T} \mathbf{R} $ and $ \mathbf{S} \mathbf{R} \mathbf{T} $ also have identical ranges. Note that the $ 3 \times 3 $ principle sub-matrix  of these two classes are $ \widetilde{\mathbf{R}} \widetilde{\mathbf{S}} $ and $ \widetilde{\mathbf{S}} \widetilde{\mathbf{R}} $ respectively. If we have
\begin{equation} \label{Eq:Exc}
\mathbf{P} \mathbf{D} = \mathbf{E} \mathbf{Q}, \mathbf{P} \mathbf{D} \mathbf{Q}^{\rmut} = \mathbf{E},
\end{equation}
where $\mathbf{P}$, $\mathbf{Q}$ are orthogonal and $\mathbf{D}$, $\mathbf{E}$ are diagonal, then using the uniqueness of singular value decomposition yields that absolute values of diagonal entries of $\mathbf{D}$ coincides that of $\mathbf{E}$ and that $ \mathbf{P}, \mathbf{Q} \neq \mathbf{I} $ iff $\mathbf{D}$ have repeated entries regardless of signs. Therefore, for most cases, \eqref{Eq:Exc} cannot be established for most cases except the one mentioned above, which means ranges of the two class do not coincide. In conclusion, a instance transformation can be achieved by $ \mathbf{T} \mathbf{R} \mathbf{S} $, $ \mathbf{R} \mathbf{T} \mathbf{S} $ and $ \mathbf{R} \mathbf{S} \mathbf{T} $ but not (always) by $ \mathbf{T} \mathbf{S} \mathbf{R} $, $ \mathbf{S} \mathbf{T} \mathbf{R} $ and $ \mathbf{S} \mathbf{R} \mathbf{T} $.)
\end{thmanswer}
\end{thmquestion}

\begin{thmquestion}
\
\begin{thmanswer}
Quaternions of the two rotations are
\begin{align}
q_x &= \frac{\sqrt{2}}{2} + \frac{\sqrt{2}}{2} \si, \\
q_y &= \frac{\sqrt{2}}{2} + \frac{\sqrt{2}}{2} \sj
\end{align}
respectively. Therefore,
\begin{equation}
\begin{split}
q_x q_y &= \rbr{ \frac{\sqrt{2}}{2} + \frac{\sqrt{2}}{2} \si } \rbr{ \frac{\sqrt{2}}{2} + \frac{\sqrt{2}}{2} \sj } \\
&= \frac{1}{2} + \frac{1}{2} \si + \frac{1}{2} \sj + \frac{1}{2} \sk.
\end{split}
\end{equation}
Therefore, composition of two rotations is a rotation about $ \rbr{ \frac{\sqrt{3}}{3}, \frac{\sqrt{3}}{3}, \frac{\sqrt{3}}{3} } $ by $ \frac{2}{3} \spi $. Moreover,
\begin{equation}
q_y q_x = \frac{1}{2} + \frac{1}{2} \si + \frac{1}{2} \sj - \frac{1}{2} \sk.
\end{equation}
\end{thmanswer}
\end{thmquestion}

\begin{thmquestion}
\
\begin{thmanswer}
According to Section 4.3.4 and Figure 4.19 of (6th edition, or Section 5.3.4 and Figure 5.19 in the 7th edition), by compositing transformations, the model view matrix should be
\begin{equation}
\mathbf{T} \rbr{\mathbf{disp}} \mathbf{R}_y \rbr{\mathit{yaw}} \mathbf{R}_x \rbr{\mathit{pitch}} \mathbf{R}_z \rbr{\mathit{roll}},
\end{equation}
where $\mathbf{disp}$ is the displacement vector from the view point to the object.
\end{thmanswer}
\end{thmquestion}

\end{document}
