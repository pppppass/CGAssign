% !TeX encoding = UTF-8
% !TeX program = LuaLaTeX
% !TeX spellcheck = en_US

% Author : Zhihan Li
% Description : Report of Assignment 4 Problem 1

\documentclass[english, nochinese]{../textmpls/pkupaper}

\usepackage[paper, tikz]{../textmpls/def}

\newcommand{\cuniversity}{Peking University}
\newcommand{\cthesisname}{Computer Graphics}
\newcommand{\titlemark}{Report of Assignment 4 Problem 1}

\DeclareRobustCommand{\authoring}%
{%
\begin{tabular}{c}%
Zhihan Li \\%
1600010653%
\end{tabular}%
}

\title{\titlemark}
\author{\authoring}

\begin{document}

\maketitle

\begin{thmquestion}
\ 
\begin{thmproof}
The setting is shown in Figure \ref{Fig:Fig}. From reflection law, $\mathbf{r}$, $\mathbf{n}$ and $\mathbf{l}$ lie in the same plane and $ \angle \rbr{ \mathbf{r}, \mathbf{n} } = \angle \rbr{ \mathbf{n}, \mathbf{l} } $. From the definition of $\mathbf{h}$, we know that $\mathbf{v}$, $\mathbf{h}$ and $\mathbf{l}$ lie in the same plane and $ \angle \rbr{ \mathbf{v}, \mathbf{h} } = \angle \rbr{ \mathbf{h}, \mathbf{l} } $.
\begin{figure}[htbp]
\centering
\begin{tikzpicture}
\filldraw (0, 0) circle [radius=2pt] node[below=3mm] {$O$};
\draw (10:1cm) arc[start angle=10, end angle=40, radius=1cm];
\draw (25:1.25cm) node {$\varphi$};
\draw (85:1cm) arc[start angle=85, end angle=100, radius=1cm];
\draw (92.5:1.5cm) node {$\psi$};
\draw[->] (0, 0) -- (10:3cm) node[right] {$\mathbf{v}$};
\draw[->] (0, 0) -- (40:3cm) node[above right] {$\mathbf{r}$};
\draw[->] (0, 0) -- (85:3cm) node[above] {$\mathbf{h}$};
\draw[->] (0, 0) -- (100:3cm) node[above] {$\mathbf{n}$};
\draw[->] (0, 0) -- (160:3cm) node[left] {$\mathbf{l}$};
\end{tikzpicture}
\caption{The angles $\phi$ and $\varphi$} \label{Fig:Fig}
\end{figure}

If $\mathbf{v}$ lies in the plane of $\mathbf{l}$, $\mathbf{n}$ and $\mathbf{r}$, then all the five vectors lie in the same plane. Therefore, we have
\begin{equation}
\begin{split}
\psi &= \angle \rbr{ \mathbf{h}, \mathbf{n} } = \angle \rbr{ \mathbf{h}, \mathbf{l} } - \angle \rbr{ \mathbf{n}, \mathbf{l} } \\
&= \frac{1}{2} \rbr{ \angle \rbr{ \mathbf{v}, \mathbf{l} } - \angle \rbr{ \mathbf{r}, \mathbf{l} } } \\
&= \frac{1}{2} \angle \rbr{ \mathbf{v}, \mathbf{r} } = \frac{1}{2} \varphi
\end{split}
\end{equation}
and equivalently $ 2 \psi = \varphi $ as desired.

\sqed
\end{thmproof}
\end{thmquestion}

\end{document}
